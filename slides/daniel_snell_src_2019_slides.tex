\documentclass[handout]{beamer}
\usetheme{Madrid}
%\mode<presentation>

\usepackage{amsfonts, amsmath, amssymb, bm}
\renewcommand{\r}{\mathbb{R}}
\newcommand{\ra}{\rightarrow}
\renewcommand{\hat}{\widehat}
\newcommand{\tract}{\mathcal{T}}
\newcommand{\confmet}{\bm{g}}

\usepackage{amssymb, amsthm,enumitem,amsmath,todonotes}

\newtheorem{theorem}{Theorem}[section]
\newtheorem{proposition}{Proposition}[section]
\newtheorem{lemma}{Lemma}[section]
\newtheorem{cor}{Corollary}[section]

\theoremstyle{definition}
\newtheorem{definition}{Definition}[section]
\newtheorem{remark}{Remark}[section]
\renewcommand{\epsilon}{\varepsilon}
\newcommand{\cupdot}{\mathbin{\mathaccent\cdot\cup}}
\newcommand{\n}{\mathbb{N}}
\newcommand{\z}{\mathbb{Z}}
\newcommand{\q}{\mathbb{Q}}
\renewcommand{\r}{\mathbb{R}}
\renewcommand{\c}{\mathbb{C}}
\newcommand{\seq}{\subset}
\newcommand{\norm}{\trianglelefteq}
\newcommand{\aut}{\textrm{Aut}}
\newcommand{\sym}{\textrm{Sym}}
\newcommand{\ra}{\rightarrow}
\newcommand{\im}{\textrm{im\,}}
\renewcommand{\ker}{\textrm{ker\,}}
\newcommand{\ex}{\backslash}
\renewcommand{\and}{\hspace{1em}\textrm{and}\hspace{1em}}

\newcommand{\tens}[3]{#1^{#2}{_{}} _{#3}}

\renewcommand{\sym}[1]{S^{#1}V^*}
\newcommand{\sk}[1]{\Lambda^{#1}V^*}
\newcommand{\st}{\,|\,}
\newcommand{\chris}[3]{\Gamma^{#1}{_{}}_{#2 #3} }
\newcommand{\udown}[3]{{#1}^{#2}{_{}}_{,#3}}

\newcommand{\set}[1]{\left\{ #1 \right\}}
\newcommand{\confmet}{\bm{g}}
\newcommand{\ric}{\textrm{Ric}}
\newcommand{\sca}{\textrm{Sc}}
\newcommand{\tract}{\mathcal{T}}
\renewcommand{\hat}[1]{\widehat{#1}}
\newcommand{\extd}{\textrm{d}}
\newcommand{\adjoint}{\mathcal{A} M}
\renewcommand{\phi}{\varphi}

\newcommand{\lpl}{
    \mbox{$
            \begin{picture}(12.7,8)(-.5,-1)
                \put(2,0.2){$+$}
                \put(6.2,2.8){\oval(8,8)[l]}
            \end{picture}$}}

\newcommand{\rpl}                         % +) or <+
{\mbox{$
            \begin{picture}(12.7,8)(-.5,-1)
                \put(0,0.2){$+$}
                \put(4.2,2.8){\oval(8,8)[r]}
            \end{picture}$}}

\newcommand{\lag}{\mathfrak{g}}
\newcommand{\lah}{\mathfrak{h}}

\newcommand{\gr}{\mathrm{gr}}

%Legacy commands; leaving these so that old versions of things will still compile. Try not to redefine these
\renewcommand{\n}{\mathbb{N}}
\renewcommand{\z}{\mathbb{Z}}
\renewcommand{\q}{\mathbb{Q}}
\renewcommand{\r}{\mathbb{R}}
\renewcommand{\c}{\mathbb{C}}

%Trying to swicth to a new convention
\newcommand{\NN}{\mathbb{N}}
\newcommand{\ZZ}{\mathbb{Z}}
\newcommand{\QQ}{\mathbb{Q}}
\newcommand{\RR}{\mathbb{R}}
\newcommand{\CC}{\mathbb{C}}
\newcommand{\id}{\mathrm{id}}


\title{CR geometry and its distinguished curves}
\author[Daniel Snell]{Daniel Snell\\{ \bigskip\bigskip \small Joint work with
A.\ Rod Gover}}
\date{\today}

\begin{document}

\begin{frame}
  \titlepage
\end{frame}

\section{Introduction}
\begin{frame}{Introduction - what is CR geometry?}
  A CR (Cauchy-Riemann or Complex-Real) manifold is a generalization of a real
  hypersurface in complex $n$-space.
  \pause
  It consists of the following data:
  \begin{itemize}
    \item a smooth \emph{real} manifold $M$ of dimension $2n+1$ and
    \item a rank $n$ \emph{complex} subbundle $H \subseteq TM$.
  \end{itemize}
  In fact, the above defines an \emph{almost CR manifold}; it is a CR manifold
  if in addition the distinguished subbundle $H$ is involutive, i.e. $[H,H]
  \subseteq H$.
  \pause

  We write $J : H \to H$ for the complex structure on the distinguished subbundle. 

  This complex structure is equivalent to a splitting of the subbundle $H_{\CC}
  \subseteq T_{\CC}M$ into a direct sub of eigenbundles: $H^{1,0}$ and
  $H^{0,1}$ which are the $+i$ and $-i$ eigenbundles of $J$ respectively.

  \pause 
  Finally, there is a Hermitian 

\end{frame}

\begin{frame}{Introduction - Parabolic geometry}
  CR manifolds are one example from a class of geometries known as
  \emph{parabolic geometries}.
  They are modelled on the coset spaces $G/P$ where $G$ is a semisimple Lie
  group and $P$ is a parabolic subgroup.

  Additionaly, $\mathfrak{g}$, the Lie algebra of $G$, comes with a symmetric
  grading.
  In the CR geometry setting, the Lie algebra is 2-graded, meaning the grading
  takes the form
  \[
    \mathfrak{g} = \mathfrak{g}_{-2} \oplus \mathfrak{g}_{-1}\oplus
    \mathfrak{g}_{0}\oplus\mathfrak{g}_{1}\oplus \mathfrak{g}_{2}.
  \]
  The subgroup $P$ has Lie algebra $\mathfrak{p} =
  \mathfrak{g}_{0}\oplus\mathfrak{g}_{1}\oplus \mathfrak{g}_{2}$.
  The grading is compatible with the grading in the sense that
  \[
    [\mathfrak{g}_\alpha, \mathfrak{g}_\beta] \subseteq \mathfrak{g}_{\alpha +
    \beta}.
  \]
  The structure of a Parabolic geometry is strongly governed by the structure
  of the Lie algebra and this grading.
\end{frame}

\begin{frame}{Introduction - Parabolic geometry}
  The tangent bundle of a parabolic geometry is also realized in terms of
  the Lie algebra, with the fiber over a point being isomorphic to the quotient
  of Lie algebras 
  \[
    T_x M \cong \mathfrak{g} / \mathfrak{p} \cong
    \mathfrak{g}_{-2} \oplus \mathfrak{g}_{-1}.
  \]
  Elements from different parts of the grading essentially give rise to
  different classes of distinguished curves on the geometry.
  Of particular note for us, curves coming from elements of $\mathfrak{g}_{-2}$
  are called \emph{chains}.
\end{frame}

\section{Tractor calculus}

\begin{frame}{Tractor bundles}
  For a given parabolic geometry, one can form certain associated vector bundles
  which admit an invariant calculus similar to the tensor calculus on Riemannian manifolds.
  \pause

  These associated vector bundles are called \emph{tractor bundles}, and the
  invariant calculus one has on them is called the \emph{tractor calculus}.

  As parabolic geometries, conformal and CR 

\end{frame}

\section{CR geometry - origins}

\begin{frame}{Fefferman's original construction}
  Fefferman was originally motivated by understanding asymptotics of
  solutions to the heat equation.

  In his original paper, given a CR manifold $M$ of hypersurface type, Fefferman
  constructs another manifold $\tilde{M}$ which is the total space of a principal
  $U(1)$-bundle over the original manifold.

  Crucially, $\tilde{M}$ carries a conformal structure which is entirely
  determined by the CR structure on $M$.
  The fibers of this bundle are null conformal 
\end{frame}

\begin{frame}{Tractor version of the original construction}
  Fefferman's original construction admits an equivalent description in terms of
  tractor calculus.

  Moreover, the (conformal) tractor bundle of the Fefferman space is related to
  the (CR) tractor bundle of the underlying manifold.

  In fact, this construction is an example of a more general phenomenon arising
  from certain special Lie group inclusions $G \hookrightarrow \tilde{G}$.

  In the (classical) Fefferman space setting, 
  \begin{itemize}
    \item $G \cong \mathrm{SU}()$ and 
    \item $\tilde{G} \cong \mathrm{SO}(2p+2, 2q+2)$.
  \end{itemize}

  \begin{theorem}
    Let
  \end{theorem}
\end{frame}

\begin{frame}{Tractor version of the original construction 2}
\end{frame}
\section{Distinguished curves and chains}

\begin{frame}{Chains}
  As has already been mentioned, chains are a class of distinguished curves
  which coming from elements of the $\mathfrak{g}_{-2}$ component of the
  grading.
\end{frame}

\end{document}
