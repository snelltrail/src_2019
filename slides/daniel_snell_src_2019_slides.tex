\documentclass{beamer}
\usetheme{Madrid}
%\mode<presentation>

\usepackage{amsfonts, amsmath, amssymb, bm}
\renewcommand{\r}{\mathbb{R}}
\newcommand{\ra}{\rightarrow}
\renewcommand{\hat}{\widehat}
\newcommand{\tract}{\mathcal{T}}
\newcommand{\confmet}{\bm{g}}
\newcommand{\CC}{\mathbb{C}}

\usepackage{amssymb, amsthm,enumitem,amsmath,todonotes}

\newtheorem{theorem}{Theorem}[section]
\newtheorem{proposition}{Proposition}[section]
\newtheorem{lemma}{Lemma}[section]
\newtheorem{cor}{Corollary}[section]

\theoremstyle{definition}
\newtheorem{definition}{Definition}[section]
\newtheorem{remark}{Remark}[section]
\renewcommand{\epsilon}{\varepsilon}
\newcommand{\cupdot}{\mathbin{\mathaccent\cdot\cup}}
\newcommand{\n}{\mathbb{N}}
\newcommand{\z}{\mathbb{Z}}
\newcommand{\q}{\mathbb{Q}}
\renewcommand{\r}{\mathbb{R}}
\renewcommand{\c}{\mathbb{C}}
\newcommand{\seq}{\subset}
\newcommand{\norm}{\trianglelefteq}
\newcommand{\aut}{\textrm{Aut}}
\newcommand{\sym}{\textrm{Sym}}
\newcommand{\ra}{\rightarrow}
\newcommand{\im}{\textrm{im\,}}
\renewcommand{\ker}{\textrm{ker\,}}
\newcommand{\ex}{\backslash}
\renewcommand{\and}{\hspace{1em}\textrm{and}\hspace{1em}}

\newcommand{\tens}[3]{#1^{#2}{_{}} _{#3}}

\renewcommand{\sym}[1]{S^{#1}V^*}
\newcommand{\sk}[1]{\Lambda^{#1}V^*}
\newcommand{\st}{\,|\,}
\newcommand{\chris}[3]{\Gamma^{#1}{_{}}_{#2 #3} }
\newcommand{\udown}[3]{{#1}^{#2}{_{}}_{,#3}}

\newcommand{\set}[1]{\left\{ #1 \right\}}
\newcommand{\confmet}{\bm{g}}
\newcommand{\ric}{\textrm{Ric}}
\newcommand{\sca}{\textrm{Sc}}
\newcommand{\tract}{\mathcal{T}}
\renewcommand{\hat}[1]{\widehat{#1}}
\newcommand{\extd}{\textrm{d}}
\newcommand{\adjoint}{\mathcal{A} M}
\renewcommand{\phi}{\varphi}

\newcommand{\lpl}{
    \mbox{$
            \begin{picture}(12.7,8)(-.5,-1)
                \put(2,0.2){$+$}
                \put(6.2,2.8){\oval(8,8)[l]}
            \end{picture}$}}

\newcommand{\rpl}                         % +) or <+
{\mbox{$
            \begin{picture}(12.7,8)(-.5,-1)
                \put(0,0.2){$+$}
                \put(4.2,2.8){\oval(8,8)[r]}
            \end{picture}$}}

\newcommand{\lag}{\mathfrak{g}}
\newcommand{\lah}{\mathfrak{h}}

\newcommand{\gr}{\mathrm{gr}}

%Legacy commands; leaving these so that old versions of things will still compile. Try not to redefine these
\renewcommand{\n}{\mathbb{N}}
\renewcommand{\z}{\mathbb{Z}}
\renewcommand{\q}{\mathbb{Q}}
\renewcommand{\r}{\mathbb{R}}
\renewcommand{\c}{\mathbb{C}}

%Trying to swicth to a new convention
\newcommand{\NN}{\mathbb{N}}
\newcommand{\ZZ}{\mathbb{Z}}
\newcommand{\QQ}{\mathbb{Q}}
\newcommand{\RR}{\mathbb{R}}
\newcommand{\CC}{\mathbb{C}}
\newcommand{\id}{\mathrm{id}}


\title{CR geometry and its distinguished curves}
\author[Daniel Snell]{Daniel Snell\\{ \bigskip\bigskip \small Joint work with
A.\ Rod Gover}}
\date{\today}

\begin{document}

\begin{frame}
  \titlepage
\end{frame}

\section{Introduction}
\begin{frame}{Introduction - what is CR geometry?}
  A CR (Cauchy-Riemann or Complex-Real) manifold is a generalization of a real
  hypersurface in complex space.
  \vfill
  \pause
  It consists of the following data:
  \begin{itemize}
    \item a smooth \emph{real} manifold $M$ of dimension $2n+1$ and
      \pause
    \item a rank $n$ \emph{complex} subbundle $H \subseteq TM$.
  \end{itemize}
  %In fact, the above defines an \emph{almost CR manifold}; it is a CR manifold
  %if in addition the distinguished subbundle $H$ is involutive, i.e. $[H,H]
  %\subseteq H$.
  \pause
  \vfill
  We write $J : H \to H$ for the complex structure on the distinguished subbundle. 

  %This complex structure is equivalent to a splitting of the complexified subbundle $H_{\CC}
  %\subseteq T_{\CC}M$
  %\[
  %  H_\CC = H^{1,0} \oplus H^{0,1}.
  %\]
  \vfill
  \pause 
  The subbundle $H$ is actually a contact distribution, so it can be realized as the kernel
  of a 1-form, called the \emph{contact form}, which we will denote $\theta$.
  There is freedom to rescale the contact form similar to conformal geometry.
  %Finally, there is a Hermitian 

\end{frame}

\begin{frame}{Introduction - parabolic geometry}
  CR manifolds are one example from a class of geometries known as
  \emph{parabolic geometries}.
  They are modeled on the coset space $G/P$ where $G$ is a semisimple Lie
  group and $P \subset G$ is a parabolic subgroup.
  \pause
  \vfill

  Additionally, $\mathfrak{g}$, the Lie algebra of $G$, comes with a decomposition
  \[
    \mathfrak{g} = \mathfrak{g}_{-2} \oplus \mathfrak{g}_{-1}\oplus
    \mathfrak{g}_{0}\oplus\mathfrak{g}_{1}\oplus \mathfrak{g}_{2}.
  \]
  The subgroup $P$ has Lie algebra $\mathfrak{p} :=
  \mathfrak{g}_{0}\oplus\mathfrak{g}_{1}\oplus \mathfrak{g}_{2}$.
  %The bracket is compatible with the grading in the sense that
  %\[
  %  [\mathfrak{g}_a, \mathfrak{g}_b] \subseteq \mathfrak{g}_{a +
  %  b}.
  %\]
  %TODO: compatibility between grading and bracket
  \pause
  \vfill
  The structure of a Parabolic geometry is strongly governed by the structure
  of the Lie algebra and this grading.
\end{frame}

\begin{frame}{Introduction - parabolic geometry and distinguished curves}
  The tangent bundle of a parabolic geometry can also be expressed in terms of
  the Lie algebra, with the fiber over a point $x \in M$ being isomorphic to the quotient
  \vfill
  \[
    T_x M \cong \mathfrak{g} / \mathfrak{p} \cong
    \mathfrak{g}_{-2} \oplus \mathfrak{g}_{-1}.
  \]
  \pause
  \vfill
  Elements from different negative parts of the grading essentially give rise to
  different classes of distinguished curves on the geometry.
  \vfill
  Of particular note, curves coming from elements of $\mathfrak{g}_{-2}$
  are called \emph{chains}.
  Chains are by far the most studied of the distinguished curves available in
  CR geometry, originating with work of Chern, Moser and Fefferman. 
\end{frame}

\section{Tractor calculus}

\begin{frame}{Tractor bundles}
  For a given parabolic geometry, one can form certain associated vector bundles
  which lead to an invariant calculus: the \emph{tractor bundles} and their \emph{tractor calculus}.
  \vfill
  \pause
  %These associated vector bundles are called \emph{tractor bundles}, and the
  %invariant calculus one has on them is called the \emph{tractor calculus}.

  %\vfill
  %As parabolic geometries, tractor bundles may be constructed for conformal and
  %CR manifolds.

  %\vfill
  For example, in the case of the CR (co)tractor bundle, after a choice of contact form
  $\theta$, one obtains the isomorphism
  \[
    \mathcal{E}_\Phi \overset{\theta}{\cong} \mathcal{E} (1,0) \oplus \mathcal{E}_\alpha (1,0)
    \oplus \mathcal{E} (0,-1).
  \]
  \vfill
  \pause
  There is also distinguished invariant bundle map
  realizing the inclusion $\mathcal{E}(0,-1) \hookrightarrow \mathcal{E}_\Phi$.
  This is called the \emph{canonical} or \emph{position} tractor, and will be
  denoted by $Z_\Phi$.

  \vfill
  The CR tractor bundle also carries a signature $(p+1,q+1)$ Hermitian bundle
  metric $h_{\Phi \overline{\Psi}}$, and a connection which preserves this metric.

\end{frame}

\section{CR geometry - origins}

\begin{frame}{Fefferman's original construction}
  Fefferman's original motivation was to understand the Bergmann kernel and
  Monge-Amp\`{e}re equations.
  \vfill
  \pause
  In his original paper, given a CR manifold $M$, Fefferman
  constructed another manifold $\tilde{M}$ which is the total space of a principal
  $U(1)$-bundle over the original manifold.
  \vfill
  \pause
  Additionally, $\tilde{M}$ carries a conformal structure which is entirely
  determined by the CR structure on $M$.
  %The fiber of the bundle $\tilde{M} \to M$ is a null conformal Killing vector field for all
  %metrics in the conformal class on $\tilde{M}$.
\end{frame}

\begin{frame}{Tractor version of the original construction}
  This \emph{Fefferman space} admits an equivalent description in the
  language of tractor calculus.
  \vfill
  %The (conformal) tractor bundle of the Fefferman space is related to
  %the (CR) tractor bundle of the underlying manifold in a very concrete way.
  %\vfill
  In fact, the entirety of the CR tractor calculus may be described in terms of the (complexified) conformal tractor calculus of its Fefferman space.
  \vfill 
  \pause
  An example of the kind of correspondence one has is the following:
  \vfill
  \begin{theorem}[\v{C}ap-Gover, 2008]
    The holomorphic holomorphic part $\tilde{Z}^\Phi$ of $X^A \in \left(\Gamma(\tilde{\mathcal{T}}_\CC [-1]\right)$ lies in the subspace $\Gamma(\mathcal{T}(1,0))$, and coincides with the section $Z^\Phi$ mentioned above (the CR canonical tractor).
  \end{theorem}
  %TODO: mention that one has similar results for other tractor projectors
\end{frame}

\section{Distinguished curves}

\begin{frame}{Distinguished curves}
  \begin{theorem}[Gover-S, 2019]
    Let $(M,H)$ be a CR manifold and $\gamma$ a curve in $M$. 
    Suppose moreover that the tangent vector of $\gamma$ always lies
    completely in the contact distribution $H$, and that $\gamma$ is null with
    respect to the Levi form. 
    The above is equivalent to the existence of a CR 2-tractor $\Sigma^{\Phi\Psi}$ which
    is parallel along $\gamma$ and satisfies the incidence relation $Z \wedge
    \Sigma = 0$, where $Z$ is the CR position tractor.
  \end{theorem}
  \vfill
  \pause
  This looks very similar to the analogous statements we already
  have characterizing distinguished curves in conformal and projective geometry!

\end{frame}

\begin{frame}{Distinguished curves}
  %As has already been mentioned, chains are a class of distinguished curves
  %which come from elements of the $\mathfrak{g}_{-2}$ component of the Lie algebra. 
  While chains can be understood via the Lie algebra, classically chains were known to be the projections of null geodesics in the Fefferman space. 
  Our existing theory of distinguished curves in conformal geometry completely describes such curves!
  \vfill
  \pause
  Our current project involves taking equations for conformal curves on the Fefferman space expressed via conformal tractors, and investigating how they descend to the CR manifold.
  \vfill
  This should give a new description of the classically studied chains, as well as new classes of CR distinguished curves.
  \vfill
  \pause
  Note that this approach is not limited to distinguished curves, but will work for any conformally invariant equation!
  %TODO mention BGG
\end{frame}

\begin{frame}{Future directions}
  \begin{itemize}
    \item There are other classes of CR distinguished curves which come from
      different components of the grading.
    \vfill
      \pause
    \item In fact, this construction is an example of a more general phenomenon arising
  from certain special Lie group inclusions $G \hookrightarrow \tilde{G}$, with
  \begin{itemize}
    \item $G \cong \mathrm{SU}(p,q)$ and 
    \item $\tilde{G} \cong \mathrm{SO}(2p+2, 2q+2)$.
  \end{itemize}
  in the (classical) Fefferman space setting.

  This should allow the distinguished curves of other geometries admitting a
  Fefferman-type construction to be described in a similar way, e.g. c-projective
  geometry.
  \vfill
  \end{itemize}
\end{frame}

\end{document}
