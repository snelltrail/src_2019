\documentclass[handout]{beamer}
\usetheme{Madrid}
%\mode<presentation>

\usepackage{amsfonts, amsmath, amssymb, bm}
\renewcommand{\r}{\mathbb{R}}
\newcommand{\ra}{\rightarrow}
\renewcommand{\hat}{\widehat}
\newcommand{\tract}{\mathcal{T}}
\newcommand{\confmet}{\bm{g}}
\newcommand{\CC}{\mathbb{C}}

\usepackage{amssymb, amsthm,enumitem,amsmath,todonotes}

\newtheorem{theorem}{Theorem}[section]
\newtheorem{proposition}{Proposition}[section]
\newtheorem{lemma}{Lemma}[section]
\newtheorem{cor}{Corollary}[section]

\theoremstyle{definition}
\newtheorem{definition}{Definition}[section]
\newtheorem{remark}{Remark}[section]
\renewcommand{\epsilon}{\varepsilon}
\newcommand{\cupdot}{\mathbin{\mathaccent\cdot\cup}}
\newcommand{\n}{\mathbb{N}}
\newcommand{\z}{\mathbb{Z}}
\newcommand{\q}{\mathbb{Q}}
\renewcommand{\r}{\mathbb{R}}
\renewcommand{\c}{\mathbb{C}}
\newcommand{\seq}{\subset}
\newcommand{\norm}{\trianglelefteq}
\newcommand{\aut}{\textrm{Aut}}
\newcommand{\sym}{\textrm{Sym}}
\newcommand{\ra}{\rightarrow}
\newcommand{\im}{\textrm{im\,}}
\renewcommand{\ker}{\textrm{ker\,}}
\newcommand{\ex}{\backslash}
\renewcommand{\and}{\hspace{1em}\textrm{and}\hspace{1em}}

\newcommand{\tens}[3]{#1^{#2}{_{}} _{#3}}

\renewcommand{\sym}[1]{S^{#1}V^*}
\newcommand{\sk}[1]{\Lambda^{#1}V^*}
\newcommand{\st}{\,|\,}
\newcommand{\chris}[3]{\Gamma^{#1}{_{}}_{#2 #3} }
\newcommand{\udown}[3]{{#1}^{#2}{_{}}_{,#3}}

\newcommand{\set}[1]{\left\{ #1 \right\}}
\newcommand{\confmet}{\bm{g}}
\newcommand{\ric}{\textrm{Ric}}
\newcommand{\sca}{\textrm{Sc}}
\newcommand{\tract}{\mathcal{T}}
\renewcommand{\hat}[1]{\widehat{#1}}
\newcommand{\extd}{\textrm{d}}
\newcommand{\adjoint}{\mathcal{A} M}
\renewcommand{\phi}{\varphi}

\newcommand{\lpl}{
    \mbox{$
            \begin{picture}(12.7,8)(-.5,-1)
                \put(2,0.2){$+$}
                \put(6.2,2.8){\oval(8,8)[l]}
            \end{picture}$}}

\newcommand{\rpl}                         % +) or <+
{\mbox{$
            \begin{picture}(12.7,8)(-.5,-1)
                \put(0,0.2){$+$}
                \put(4.2,2.8){\oval(8,8)[r]}
            \end{picture}$}}

\newcommand{\lag}{\mathfrak{g}}
\newcommand{\lah}{\mathfrak{h}}

\newcommand{\gr}{\mathrm{gr}}

%Legacy commands; leaving these so that old versions of things will still compile. Try not to redefine these
\renewcommand{\n}{\mathbb{N}}
\renewcommand{\z}{\mathbb{Z}}
\renewcommand{\q}{\mathbb{Q}}
\renewcommand{\r}{\mathbb{R}}
\renewcommand{\c}{\mathbb{C}}

%Trying to swicth to a new convention
\newcommand{\NN}{\mathbb{N}}
\newcommand{\ZZ}{\mathbb{Z}}
\newcommand{\QQ}{\mathbb{Q}}
\newcommand{\RR}{\mathbb{R}}
\newcommand{\CC}{\mathbb{C}}
\newcommand{\id}{\mathrm{id}}


\title{CR geometry and its distinguished curves}
\author[Daniel Snell]{Daniel Snell\\{ \bigskip\bigskip \small Joint work with
A.\ Rod Gover}}
\date{\today}

\begin{document}

\begin{frame}
  \titlepage
\end{frame}

\section{Introduction}
\begin{frame}{Introduction - what is CR geometry?}
  A CR (Cauchy-Riemann or Complex-Real) manifold is a generalization of a real
  hypersurface in complex $n$-space.
  \pause
  It consists of the following data:
  \begin{itemize}
    \item a smooth \emph{real} manifold $M$ of dimension $2n+1$ and
    \item a rank $n$ \emph{complex} subbundle $H \subseteq TM$.
  \end{itemize}
  %In fact, the above defines an \emph{almost CR manifold}; it is a CR manifold
  %if in addition the distinguished subbundle $H$ is involutive, i.e. $[H,H]
  %\subseteq H$.
  \pause

  We write $J : H \to H$ for the complex structure on the distinguished subbundle. 

  This complex structure is equivalent to a splitting of the complexified subbundle $H_{\CC}
  \subseteq T_{\CC}M$
  \[
    H_\CC = H^{1,0} \oplus H^{0,1}.
  \]
  \pause 
  Moreover, $H$ is a contact distribution, so it can be realized as the kernel
  of a 1-form, called the \emph{contact form}, which we will denote $\theta$.
  There is freedom to rescale the contact form similar to conformal geometry.
  %Finally, there is a Hermitian 

\end{frame}

\begin{frame}{Introduction - Parabolic geometry}
  CR manifolds are one example from a class of geometries known as
  \emph{parabolic geometries}.
  They are modelled on the coset spaces $G/P$ where $G$ is a semisimple Lie
  group and $P$ is a parabolic subgroup.

  Additionaly, $\mathfrak{g}$, the Lie algebra of $G$, comes with a symmetric
  grading.
  In the CR geometry setting, the Lie algebra is 2-graded, meaning the grading
  takes the form
  \[
    \mathfrak{g} = \mathfrak{g}_{-2} \oplus \mathfrak{g}_{-1}\oplus
    \mathfrak{g}_{0}\oplus\mathfrak{g}_{1}\oplus \mathfrak{g}_{2}.
  \]
  The subgroup $P$ has Lie algebra $\mathfrak{p} =
  \mathfrak{g}_{0}\oplus\mathfrak{g}_{1}\oplus \mathfrak{g}_{2}$.
  The bracket is compatible with the grading in the sense that
  \[
    [\mathfrak{g}_\alpha, \mathfrak{g}_\beta] \subseteq \mathfrak{g}_{\alpha +
    \beta}.
  \]
  The structure of a Parabolic geometry is strongly governed by the structure
  of the Lie algebra and this grading.
\end{frame}

\begin{frame}{Introduction - Parabolic geometry and distinguished curves}
  The tangent bundle of a parabolic geometry is also realized in terms of
  the Lie algebra, with the fiber over a point $x \in M$ being isomorphic to the quotient
  of Lie algebras 
  \[
    T_x M \cong \mathfrak{g} / \mathfrak{p} \cong
    \mathfrak{g}_{-2} \oplus \mathfrak{g}_{-1}.
  \]
  Elements from different negative parts of the grading essentially give rise to
  different classes of distinguished curves on the geometry.

  Of particular note, curves coming from elements of $\mathfrak{g}_{-2}$
  are called \emph{chains}.
  Chains are by far the most studied of the distinguished curves available in
  CR geometry, originating with work of Chern, Moser and Fefferman. 
\end{frame}

\section{Tractor calculus}

\begin{frame}{Tractor bundles}
  For a given parabolic geometry, one can form certain associated vector bundles
  which admit an invariant calculus.
  \vfill
  These associated vector bundles are called \emph{tractor bundles}, and the
  invariant calculus one has on them is called the \emph{tractor calculus}.

  \vfill
  %As parabolic geometries, tractor bundles may be constructed for conformal and
  %CR manifolds.

  %\vfill
  In the case of the CR (co)tractor bundle, after a choice of contact form
  $\theta$, one obtains the isomorphism
  \[
    \mathcal{E}_A \overset{\theta}{\cong} \mathcal{E} (1,0) \oplus \mathcal{E}_\alpha
    \oplus \mathcal{E} (0,-1).
  \]
  \vfill
  For any parabolic geometry, there is a distinguished invariant bundle map
  realizing the inclusion $\mathcal{E}(0,-1) \hookrightarrow \mathcal{E}_A$.
  This is called the \emph{canonical} or \emph{position} tractor, and will be
  denoted by $Z_A$.

  \vfill
  The CR tractor bundle also carries a signature $(p+1,q+1)$ Hermitian bundle
  metric $h_{A \overline{B}}$, and a connection which preserves this metric.

\end{frame}

\section{CR geometry - origins}

\begin{frame}{Fefferman's original construction}
  Fefferman's original motivation was to understand the Bergmann kernel and
  Monge-Amp\`{e}re equations.
  \vfill
  In his original paper, given a CR manifold $M$ of hypersurface type, Fefferman
  constructs another manifold $\tilde{M}$ which is the total space of a principal
  $U(1)$-bundle over the original manifold.
  \vfill
  Crucially, $\tilde{M}$ carries a conformal structure which is entirely
  determined by the CR structure on $M$.
  The fibers of the bundle $\tilde{M} \to M$ are null conformal Killing vector fields for all
  metrics in the conformal class on $\tilde{M}$.
\end{frame}

\begin{frame}{Tractor version of the original construction}
  Fefferman's original construction admits an equivalent description in the
  language of tractor calculus.

  The (conformal) tractor bundle of the Fefferman space is related to
  the (CR) tractor bundle of the underlying manifold.

  

  \begin{theorem}[\v{C}ap-Gover, 2008]
    Let
  \end{theorem}
\end{frame}

\begin{frame}{Tractor version of the original construction 2}
\end{frame}
\section{Distinguished curves}

\begin{frame}{Distinguished curves}
  \begin{theorem}[Gover-S, 2018]
    Let $(M,H)$ be a CR manifold and $\gamma$ a curve in $M$. 
    Suppose moreover that the tangent vector of $\gamma$ always lies
    completely in the contact distribution $H$, and that $\gamma$ is null with
    respect to the Levi form. 
    The above is equivalent to the existence of a 2-tractor $\Sigma^{AB}$ which
    is parallel along $\gamma$ and satisfies the incidence relation $Z \wedge
    \Sigma = 0$, where $Z$ is the CR position tractor.
  \end{theorem}
  This is very similar in appearence to the analogous statements we already
  have characerizing distinguished curves in conformal and projective geometry!
  As has already been mentioned, chains are a class of distinguished curves
  which come from elements of the $\mathfrak{g}_{-2}$ component of the
  grading. 
  Moreover, these distinguished curves are equivalently known to be the
  projections of null geodesics in the Fefferman space. 
\end{frame}

\begin{frame}{Future directions}
  \begin{itemize}
    \item There are other classes of CR distinguished curves coming from
      different components of the grading.
    \vfill
    \item In fact, this construction is an example of a more general phenomenon arising
  from certain special Lie group inclusions $G \hookrightarrow \tilde{G}$, with
  \begin{itemize}
    \item $G \cong \mathrm{SU}(p,q)$ and 
    \item $\tilde{G} \cong \mathrm{SO}(2p+2, 2q+2)$.
  \end{itemize}
  in the (classical) Fefferman space setting.

  This should allow the distinguished curves of other geometries admitting a
  Fefferman-type construction to be described in a similar way, e.g. c-projective
  geometry.
  \vfill
  \end{itemize}
\end{frame}

\end{document}
